\documentclass[12pt,a4paper]{article}
\usepackage{cmap} % Makes the PDF copiable. See http://tex.stackexchange.com/a/64198/25761
\usepackage[T1]{fontenc}
\usepackage[brazil]{babel}
\usepackage[utf8]{inputenc}
\usepackage{amsmath}
\usepackage{amsfonts}
\usepackage{amssymb}
\usepackage{amsthm}
\usepackage{textcomp} % \degree
\usepackage{gensymb} % \degree
\usepackage[usenames,svgnames,dvipsnames]{xcolor}
\usepackage{hyperref}
\usepackage{multicol}
\usepackage{graphicx}
\usepackage{systeme}
\usepackage[margin=2cm]{geometry}

\hypersetup{
    colorlinks = true,
    allcolors = {blue}
}

\newcommand{\vect}[1]{\overrightarrow{#1}}
\newcommand{\norm}[1]{\left|\left|{#1}\right|\right|}

\newcommand*\tipo{PROVA III}
\newcommand*\turma{TURMA C}
\newcommand*\disciplina{GAN0001}
\newcommand*\nome{GEOMETRIA ANALÍTICA}
\newcommand*\eu{Helder G. G. de Lima}
\newcommand*\data{27/05/2015}

\author{\eu}
\title{\tipo - \disciplina}
\date{\data}

\begin{document}
\thispagestyle{empty}
\newgeometry{margin=2cm,bottom=0.5cm}
\begin{center}
\includegraphics{udesc_joinville_cabecalho.pdf}
\\ Prof. \eu\footnote{
Este é um material de acesso livre distribuído sob os termos da licença \href{https://creativecommons.org/licenses/by-sa/4.0/deed.pt_BR}{Creative Commons BY-SA 4.0}.}

\noindent\begin{tabular}{l c c r}
  \textbf{\disciplina}
& \textbf{\tipo}
& \textbf{\data}
& \textbf{\turma}
\end{tabular}\vspace{-0.3cm}
\noindent\rule{17cm}{0.01cm}
\end{center}

\noindent Nome do aluno: \rule{14cm}{0.01cm}

\section*{Instruções}

\begin{enumerate}
\renewcommand{\theenumi}{\Roman{enumi}}
\item Identifique-se em todas as folhas.
\item Mantenha o celular e os demais equipamentos eletrônicos desligados durante a prova.
\item Justifique cada resposta com cálculos ou argumentos baseados na teoria estudada.
\item Escolha \textsc{\textbf{uma}} questão para \textsc{\textbf{não}} fazer (ela não será corrigida): \rule{3cm}{0.01cm}
\end{enumerate}

\section*{Questões}

\begin{enumerate}
\item (2,0 pontos) Sejam $A=(3,3,3)$, $B=(-3,-3,3)$, $C=(3,-3,-3)$ e $D=(-3,3,-3)$. Qual é a altura do tetraedro com vértices $A$, $B$, $C$ e $D$, relativa à base $BCD$?

\item (2,0 pontos) Qual é a distância entre a reta
$r: \frac{x}{4} = \frac{y-1/2}{3} = \frac{z-3/2}{2}$ e
$s: \frac{x}{-4} = \frac{y+1/2}{6} = z+1$?

\item (2,0 pontos) Calcule a distância entre $r$ e $s$, sendo:
\[
r: \begin{cases}
y = \frac{x}{2} - 3\\
z= x - 4 \qquad
\end{cases}
s: \frac{x-1}{2} = y = \frac{z-4}{2}
\]

\item (2,0 pontos) Determine a equação, o foco e a reta diretriz da parábola cujo vértice é o ponto $(3,-1)$, cujo eixo de simetria é a reta $y+1=0$ e que passa pelo ponto $(-6,5)$.

\item (2,0 pontos) Verifique qual das seguintes elipses possui a maior excentricidade:
\begin{multicols}{2}
\begin{enumerate}
%\item $16(y+5)^2-9x^2+36x-37 = 0$
\item $9 x^2-72 x+25 y^2+50 y-56 = 0$
\item $\frac{(x-4)^2}{25} + \frac{(y+1)^2}{169} = 1$
\end{enumerate}
\end{multicols}

\item (2,0 pontos) Obtenha os focos e a equação da hipérbole que possui vértices $A_1=(-2,1)$ e $A_2 = (4,1)$, e que contém o ponto $P = (-5,4)$.
\end{enumerate}

\section*{Lembrete}
%Nas fórmulas a seguir, $P_0$, $P_1$, $P_2$ são pontos, $r$, $s$ são retas, $\vec{u}$, $\vec{v}$ são vetores e $\pi$ é um plano:
\begin{itemize}\footnotesize
\item $d(P_1, P_2) =
\sqrt{(x_2-x_1)^2 + (y_2-y_1)^2 + (z_2-z_1)^2}$
\begin{multicols}{3}
\item $d(P_0, r) =
\frac{ \norm{ \vec{v} \times \vect{P_0P_1} } }
     { \norm{\vec{v}} }$
\item $d(r, s) =
\frac{ \left| ( \vec{u}, \vec{v}, \vect{P_1P_2} ) \right| }
     { || \vec{u} \times \vec{v} || }$
\item $d(P_0, \pi) =
\frac{ \left| ax_0 + by_0 + cz_0 + d \right| }
     { \sqrt{a^2 + b^2 + c^2} }$
\item $(x-h)^2 = 2p(y-k)$
\item $(y-k)^2 = 2p(x-h)$
\item $\dfrac{(x-h)^2}{a^2} + \dfrac{(y-k)^2}{b^2} = 1$
\item $\dfrac{(x-h)^2}{b^2} + \dfrac{(y-k)^2}{a^2} = 1$
\item $\dfrac{(x-h)^2}{a^2} - \dfrac{(y-k)^2}{b^2} = 1$
\item $\dfrac{(y-k)^2}{a^2} - \dfrac{(x-h)^2}{b^2} = 1$
\end{multicols}
\end{itemize}

\newpage
\restoregeometry
\section*{Respostas e observações}
\begin{enumerate}
\item A altura do tetraedro com vértices $A=(3,3,3)$, $B=(-3,-3,3)$, $C=(3,-3,-3)$ e $D=(-3,3,-3)$, relativa à base $BCD$ é igual à distância do ponto $A$ ao plano $\pi$ definido pelos pontos $B$, $C$ e $D$. Se $P = (x,y,z)$ então $P \in \pi$ se, e somente se, $\vect{BP}=(x+3,y+3,z-3)$, $\vect{BC}=(6,0,-6)$ e $\vect{BD}=(0,6,-6)$ forem coplanares, isto é,
\[
0=\begin{vmatrix}
x+3 & y+3 & z-3\\
  6 &   0 &  -6\\
  0 &   6 &  -6
\end{vmatrix}
= 36x+36y+36z+108.
\]
Assim, tem-se $\pi: x+y+z+3=0$ e $\vec{n}=(1,1,1)$ é um vetor normal a $\pi$. Consequentemente,
\[
d(A,\pi)
= \frac{|3+3+3+3|}{\sqrt{1^2+1^2+1^2}}
= \frac{12}{\sqrt{3}}
= 4 \sqrt{3}.
\]

\item Considerando que a reta $r: \frac{x}{4} = \frac{y-1/2}{3} = \frac{z-3/2}{2}$ tem um vetor diretor $\vec{u}=(4,3,2)$ e que a reta $s: \frac{x}{-4} = \frac{y+1/2}{6} = z+1$ tem um vetor diretor $\vec{v}=(-4,6,1)$, e que nenhum deles é um múltiplo escalar do outro, conclui-se que $r$ e $s$ não são paralelas. Neste caso, tomando $P_1 = (0,1/2,3/2) \in r$ e $P_2 = (0,-1/2,-1) \in s$, a distância entre as retas é $d(r, s)
= \frac{ \left| ( \vec{u}, \vec{v}, \vect{P_1P_2} ) \right| }
     { || \vec{u} \times \vec{v} || }
$, em que
\[
( \vec{u}, \vec{v}, \vect{P_1P_2} )
= \begin{vmatrix}
 4 &  3 &  2\\
-4 &  6 &  1\\
 0 & -1 & -5/2
\end{vmatrix}
=-78
\]
e
\[
\vec{u} \times \vec{v}
= \begin{vmatrix}
\vec{i} & \vec{j} & \vec{k}\\
 4 & 3 & 2\\
-4 & 6 & 1
\end{vmatrix}
=(-9, -12, 36)
\]
Portanto,
\[
d(r, s)
= \frac{ \left| -78 \right| }
     { || (-9, -12, 36) || }
= \frac{ \left| -2\cdot 3 \cdot 13 \right| }
     { 3 \cdot || (-3, -4, 12) || }
= \frac{ 2 \cdot 13 }
     { \sqrt{9+16+144} }
=
\frac{ 2 \cdot 13 }
     { 13 }
= 2.
\]

\item Como $\vec{u}=(1,\frac{1}{2},1)$ é um vetor diretor da reta $r: \begin{cases}
y = \frac{x}{2} - 3\\
z= x - 4
\end{cases}$ e $\vec{v}=(2,1,2) = 2 \cdot \vec{u}$ é um vetor diretor da reta $s: \frac{x-1}{2} = y = \frac{z-4}{2}$, conclui-se que $r$ e $s$ são paralelas. Neste caso, tomando $P_0 = (0, -3, -4) \in r$ e $P_1 = (1, 0, 4) \in s$, tem-se
\[
d(r,s)
= d(P_0,s)
= \frac{|| \vec{v} \times \vect{P_1 P_0} ||}{|| \vec{v} ||},
\]
em que
\[
\vec{v} \times \vect{P_1 P_0}
=(2,1,2) \times (-1,-3,-8)
=\begin{vmatrix}
\vec{i} & \vec{j} & \vec{k}\\
 2 &  1 &  2\\
-1 & -3 & -8
\end{vmatrix}
= (-2, 14, -5)
\]
e
\[
|| \vec{v} || = \sqrt{2^2+1^2+2^2} = 3.
\]
Portanto,
\[
d(r,s)
= \frac{|| (-2, 14, -5) || }{3}
= \frac{\sqrt{4+196+25} }{3}
= \frac{15}{3}
= 5.
\]

\item \textit{Determine a equação, o foco e a reta diretriz da parábola cujo vértice é o ponto $(3,-1)$, cujo eixo de simetria é a reta $y+1=0$ e que passa pelo ponto $(-6,5)$.}

Como o eixo de simetria da parábola é horizontal, e ela tem vértice no ponto $V = (3,-1)$, sua equação tem a forma padrão $(y-(-1))^2 = 2p(x-3)$. Além disso, como a parábola passa pelo ponto $P=(-6,5)$, o valor de $p \in \mathbb{R}$ deve ser tal que
\[
(5-(-1))^2 = 2p(-6-3)
\Leftrightarrow
36 = -18p
\Leftrightarrow
p = -2.
\]
Assim, a equação é $(y+1)^2 = -4(x-3)$. Neste caso, como o foco está sobre o eixo de simetria $y=-1$ e sua distância ao vértice é a metade da distância $p$ entre o foco e a reta diretriz, tem-se $F=(3+\frac{p}{2},-1) = (3-1,-1) = (2,-1)$. A reta diretriz é $d: x = 3-\frac{p}{2} = 3 + 1 = 4$, ou seja, $d: x-4 = 0$.

\item A equação $9 x^2-72 x+25 y^2+50 y-56 = 0$ pode ser reescrita do seguinte modo:
\begin{align*}
9 x^2-72 x+25 y^2+50 y-56 = 0
& \Leftrightarrow
9 (x^2 - 8x) + 25(y^2 +2y) - 56 = 0\\
& \Leftrightarrow
9 ((x-4)^2 - 16) + 25((y+1)^2 -1) - 56 = 0\\
& \Leftrightarrow
9(x-4)^2 + 25(y+1)^2 = 225\\
& \Leftrightarrow
\frac{(x-4)^2}{25} + \frac{(y+1)^2}{9} = 1
\end{align*}
Assim, para esta elipse, $a=5$ e $c=\sqrt{25-9}=4$, de modo que sua excentricidade é $e_1 = \frac{4}{5} = 0,8$.

Por outro lado, no caso da elipse $\frac{(x-4)^2}{25} + \frac{(y+1)^2}{169} = 1$, tem-se $a = 13$ e $c=\sqrt{169-15} = 12$, e portanto sua excentricidade é $e_2 = \frac{12}{13} \approx 0,9 > 0,8 = e_1$.

\item Como o centro $C$ da hipérbole é o ponto médio dos vértices, tem-se
\[
C = \left(\frac{-2+4}{2},\frac{1+1}{2}\right) = (1,1).
\]
Além disso, $2a = d(A_1,A_2) = \sqrt{(4-(-2))^2+(1-1)^2} = 6$, ou seja, $a = 3$ e como os vértices estão sobre uma reta horizontal, a equação terá a forma padrão $\frac{(x-1)^2}{3^2}-\frac{(y-1)^2}{b^2}=1$. Como a hipérbole passa por $P=(-5,4)$, tem-se
\[
\frac{(-5-1)^2}{3^2}-\frac{(4-1)^2}{b^2}=1
\Leftrightarrow
\frac{36}{9}-\frac{9}{b^2}=1
\Leftrightarrow
b^2=3
\]

Portanto a hipérbole tem equação $\frac{(x-1)^2}{9} + \frac{(y-1)^2}{3} = 1$.

Finalmente, como $c = \sqrt{a^2+b^2}$, tem-se $c = \sqrt{9+3} = \sqrt{12}= 2\sqrt{3}$ e consequentemente os focos são
$F_1 = (1+2\sqrt{3},1)$ e
$F_2 = (1-2\sqrt{3},1)$.



\end{enumerate}

\end{document}
