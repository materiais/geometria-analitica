\documentclass[12pt,a4paper]{article}
\usepackage{cmap} % Makes the PDF copiable. See http://tex.stackexchange.com/a/64198/25761
\usepackage[T1]{fontenc}
\usepackage[brazil]{babel}
\usepackage[utf8]{inputenc}
\usepackage{amsmath}
\usepackage{amsfonts}
\usepackage{amssymb}
\usepackage{amsthm}
\usepackage{textcomp} % \degree
\usepackage{gensymb} % \degree
\usepackage[usenames,svgnames,dvipsnames]{xcolor}
\usepackage{hyperref}
\usepackage{multicol}
\usepackage{graphicx}
\usepackage{systeme}
\usepackage[margin=2cm]{geometry}

\hypersetup{
    colorlinks = true,
    allcolors = {blue}
}

\newcommand{\vect}[1]{\overrightarrow{#1}}
\newcommand{\norm}[1]{\left|\left|{#1}\right|\right|}

\newcommand*\tipo{PROVA III}
\newcommand*\turma{TURMA C}
\newcommand*\disciplina{GAN0001}
\newcommand*\nome{GEOMETRIA ANALÍTICA}
\newcommand*\eu{Helder G. G. de Lima}
\newcommand*\data{27/05/2015}

\author{\eu}
\title{\tipo - \disciplina}
\date{\data}

\begin{document}
\thispagestyle{empty}
\newgeometry{margin=2cm,bottom=0.5cm}
\begin{center}
\includegraphics{udesc_joinville_cabecalho.pdf}
\\ Prof. \eu\footnote{
Este é um material de acesso livre distribuído sob os termos da licença \href{https://creativecommons.org/licenses/by-sa/4.0/deed.pt_BR}{Creative Commons BY-SA 4.0}.}

\noindent\begin{tabular}{l c c r}
  \textbf{\disciplina}
& \textbf{\tipo}
& \textbf{\data}
& \textbf{\turma}
\end{tabular}\vspace{-0.3cm}
\noindent\rule{17cm}{0.01cm}
\end{center}

\noindent Nome do aluno: \rule{14cm}{0.01cm}

\section*{Instruções}

\begin{enumerate}
\renewcommand{\theenumi}{\Roman{enumi}}
\item Identifique-se em todas as folhas.
\item Mantenha o celular e os demais equipamentos eletrônicos desligados durante a prova.
\item Justifique cada resposta com cálculos ou argumentos baseados na teoria estudada.
\item Escolha \textsc{\textbf{uma}} questão para \textsc{\textbf{não}} fazer (ela não será corrigida): \rule{3cm}{0.01cm}
\end{enumerate}

\section*{Questões}

\begin{enumerate}
\item (2,0 pontos) Sejam $A=(3,3,3)$, $B=(-3,-3,3)$, $C=(3,-3,-3)$ e $D=(-3,3,-3)$. Qual é a altura do tetraedro com vértices $A$, $B$, $C$ e $D$, relativa à base $BCD$?

\item (2,0 pontos) Qual é a distância entre a reta
$r: \frac{x}{4} = \frac{y-1/2}{3} = \frac{z-3/2}{2}$ e
$s: \frac{x}{-4} = \frac{y+1/2}{6} = z+1$?

\item (2,0 pontos) Calcule a distância entre $r$ e $s$, sendo:
\[
r: \begin{cases}
y = \frac{x}{2} - 3\\
z= x - 4 \qquad
\end{cases}
s: \frac{x-1}{2} = y = \frac{z-4}{2}
\]

\item (2,0 pontos) Determine a equação, o foco e a reta diretriz da parábola cujo vértice é o ponto $(3,-1)$, cujo eixo de simetria é a reta $y+1=0$ e que passa pelo ponto $(-6,5)$.

\item (2,0 pontos) Verifique qual das seguintes elipses possui a maior excentricidade:
\begin{multicols}{2}
\begin{enumerate}
%\item $16(y+5)^2-9x^2+36x-37 = 0$
\item $9 x^2-72 x+25 y^2+50 y-56 = 0$
\item $\frac{(x-4)^2}{25} + \frac{(y+1)^2}{169} = 1$
\end{enumerate}
\end{multicols}

\item (2,0 pontos) Obtenha os focos e a equação da hipérbole que possui vértices $A_1=(-2,1)$ e $A_2 = (4,1)$, e que contém o ponto $P = (-5,4)$.
\end{enumerate}

\section*{Lembrete}
%Nas fórmulas a seguir, $P_0$, $P_1$, $P_2$ são pontos, $r$, $s$ são retas, $\vec{u}$, $\vec{v}$ são vetores e $\pi$ é um plano:
\begin{itemize}\footnotesize
\item $d(P_1, P_2) =
\sqrt{(x_2-x_1)^2 + (y_2-y_1)^2 + (z_2-z_1)^2}$
\begin{multicols}{3}
\item $d(P_0, r) =
\frac{ \norm{ \vec{v} \times \vect{P_0P_1} } }
     { \norm{\vec{v}} }$
\item $d(r, s) =
\frac{ \left| ( \vec{u}, \vec{v}, \vect{P_1P_2} ) \right| }
     { || \vec{u} \times \vec{v} || }$
\item $d(P_0, \pi) =
\frac{ \left| ax_0 + by_0 + cz_0 + d \right| }
     { \sqrt{a^2 + b^2 + c^2} }$
\item $(x-h)^2 = 2p(y-k)$
\item $(y-k)^2 = 2p(x-h)$
\item $\dfrac{(x-h)^2}{a^2} + \dfrac{(y-k)^2}{b^2} = 1$
\item $\dfrac{(x-h)^2}{b^2} + \dfrac{(y-k)^2}{a^2} = 1$
\item $\dfrac{(x-h)^2}{a^2} - \dfrac{(y-k)^2}{b^2} = 1$
\item $\dfrac{(y-k)^2}{a^2} - \dfrac{(x-h)^2}{b^2} = 1$
\end{multicols}
\end{itemize}

%\newpage
\restoregeometry
%\section*{Respostas e observações}
\end{document}
