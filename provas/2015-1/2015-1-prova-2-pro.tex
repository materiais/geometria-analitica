\documentclass[12pt,a4paper]{article}
\usepackage{cmap} % Makes the PDF copiable. See http://tex.stackexchange.com/a/64198/25761
\usepackage[T1]{fontenc}
\usepackage[brazil]{babel}
\usepackage[utf8]{inputenc}
\usepackage{amsmath}
\usepackage{amsfonts}
\usepackage{amssymb}
\usepackage{amsthm}
\usepackage{textcomp} % \degree
\usepackage{gensymb} % \degree
\usepackage[usenames,svgnames,dvipsnames]{xcolor}
\usepackage{hyperref}
\usepackage{multicol}
\usepackage{graphicx}
\usepackage{systeme}
\usepackage[margin=2cm]{geometry}

\hypersetup{
    colorlinks = true,
    allcolors = {blue}
}

\newcommand{\vect}[1]{\overrightarrow{#1}}
\newcommand{\norm}[1]{\left|\left|{#1}\right|\right|}

\newcommand*\tipo{PROVA II}
\newcommand*\turma{TURMA C}
\newcommand*\disciplina{GAN0001}
\newcommand*\nome{GEOMETRIA ANALÍTICA}
\newcommand*\eu{Helder G. G. de Lima}
\newcommand*\data{29/04/2015}

\author{\eu}
\title{\tipo - \disciplina}
\date{\data}

\begin{document}
\thispagestyle{empty}
\newgeometry{margin=2cm,bottom=0.5cm}
\begin{center}
\includegraphics{udesc_joinville_cabecalho.pdf}
\\ Prof. \eu\footnote{
Este é um material de acesso livre distribuído sob os termos da licença \href{https://creativecommons.org/licenses/by-sa/4.0/deed.pt_BR}{Creative Commons BY-SA 4.0}.}

\noindent\begin{tabular}{l c c r}
  \textbf{\disciplina}
& \textbf{\tipo}
& \textbf{\data}
& \textbf{\turma}
\end{tabular}\vspace{-0.3cm}
\noindent\rule{17cm}{0.01cm}
\end{center}

\noindent Nome do aluno: \rule{14cm}{0.01cm}

\section*{Instruções}

\begin{enumerate}
\renewcommand{\theenumi}{\Roman{enumi}}
\item Identifique-se em todas as folhas.
\item Mantenha o celular e os demais equipamentos eletrônicos desligados durante a prova.
\item Justifique cada resposta com cálculos ou argumentos baseados na teoria estudada.
\item Escolha \textsc{\textbf{uma}} questão para \textsc{\textbf{não}} fazer (ela não será corrigida): \rule{3cm}{0.01cm}
\end{enumerate}

\section*{Questões}

\begin{enumerate}
\item (2,0 pontos) Dados $\vec{u}=(0,1,2)$, $\vec{v}=(0,-1,2)$ e $\vec{w}=(-2,0,1)$, calcule
\[
  \vec{u} \times (\vec{v} \times \vec{w}) \, + \,
  \vec{v} \times (\vec{w} \times \vec{u}) \, + \,
  \vec{w} \times (\vec{u} \times \vec{v}).
\]
\item (2,0 pontos) Determine as equações reduzidas (com variável independente $y$) da reta $r$ que passa por $A = (1,2,-2)$ e que é perpendicular ao plano $-\frac{x}{2} + y - 3z - 5 = 0$.

\item (2,0 pontos) Encontre equações paramétricas da reta $r$ dada pela interseção do plano $-3x + 6y - 4z + 12 = 0$ com o plano $-x + 2y + 6z + 4 = 0$.


\item (2,0 pontos) Calcule o ângulo entre a reta $r: x+2 = \dfrac{y-3}{-2} = z$ e a reta $s : \begin{cases}
x = -y -1 \\
z = 2.
\end{cases}$

\item (2,0 pontos) Forneça uma equação para o plano que passa pelo ponto médio de $A = (2, 1, 3)$ e $B = (0, 3, -9)$, e que além disso contém a reta $x = y = z$.

\item (2,0 pontos) Seja $s$ a reta $\frac{x + 2}{2} = \frac{y - 3}{-3} = z$. Dê um exemplo (com equações paramétricas) de uma reta $r_1 // s$, de uma reta $r_2$ concorrente com $s$ e de uma reta $r_3$ reversa com $s$ (explique suas escolhas).
\end{enumerate}

\section*{Lembrete}
Se $\vec{u} = (a,b,c)$ então $\norm{\vec{u}} = \sqrt{a^2 + b^2 + c^2}$. Dados $\vec{v} = (a_1, b_1, c_1)$ e $\vec{w} = (a_2, b_2, c_2)$, tem-se:
%\begin{multicols}{2}
\begin{itemize}
\item $\vec{v} \cdot \vec{w}
= a_1 a_2 + b_1 b_2 + c_1 c_2
= \norm{\vec{v}} \cdot \norm{\vec{w}} \cdot \cos(\theta)
$, onde $\theta$ é o ângulo entre os vetores $\vec{v}$ e $\vec{w}$.
\item $\vec{v} \times \vec{w}
= \begin{vmatrix}
\vec{i} & \vec{j} & \vec{k} \\
a_1 & b_1 & c_1\\
a_2 & b_2 & c_2
\end{vmatrix}$
\item $(\vec{u}, \vec{v}, \vec{w})
= \vec{u} \cdot \vec{v} \times \vec{w}
= \begin{vmatrix}
a   & b   & c \\
a_1 & b_1 & c_1 \\
a_2 & b_2 & c_2
\end{vmatrix}$
\end{itemize}
%\end{multicols}
\newpage
\restoregeometry
\section*{Respostas e observações}

\begin{enumerate}
\item \textit{Dados $\vec{u}=(0,1,2)$, $\vec{v}=(0,-1,2)$ e $\vec{w}=(-2,0,1)$, calcule
\[
  \vec{u} \times (\vec{v} \times \vec{w}) \, + \,
  \vec{v} \times (\vec{w} \times \vec{u}) \, + \,
  \vec{w} \times (\vec{u} \times \vec{v}).
\]
}
Levando em conta que $\vec{u} \times (\vec{v} \times \vec{w})
= \begin{vmatrix}
\vec{v} & \vec{w} \\
\vec{u} \cdot \vec{v} & \vec{u} \cdot \vec{w}
\end{vmatrix}$, para quaisquer $\vec{u}, \vec{v}$ e $\vec{w}$, podemos calcular primeiramente:
\[
\vec{u} \cdot \vec{v} = 0 \cdot 0 + 1 \cdot (-1) + 2 \cdot 2 = 3
\]
\[
\vec{u} \cdot \vec{w} = 0 \cdot (-2) + 1 \cdot 0 + 2 \cdot 1 = 2
\]
\[
\vec{v} \cdot \vec{w} = 0 \cdot (-2) + (-1) \cdot 0 + 2 \cdot 1 = 2
\]
e então substituir:
\[
\vec{u} \times (\vec{v} \times \vec{w})
= \begin{vmatrix}
\vec{v} & \vec{w} \\
3 & 2
\end{vmatrix}
= 2\vec{v} -3 \vec{w}
\]
\[
\vec{v} \times (\vec{w} \times \vec{u})
= \begin{vmatrix}
\vec{w} & \vec{u} \\
2 & 3
\end{vmatrix}
= 3\vec{w} -2 \vec{u}
\]
\[
\vec{w} \times (\vec{u} \times \vec{v})
= \begin{vmatrix}
\vec{u} & \vec{v} \\
2 & 2
\end{vmatrix}
= 2\vec{u} -2 \vec{v}
\]
Assim, a soma procurada é $(2\vec{v} -3 \vec{w}) + (3\vec{w} -2 \vec{u}) + (2\vec{u} -2 \vec{v}) = \vec{0}$.

\textbf{Nota:} também é possível expandir expressões como $2\vec{v} -3 \vec{w} = 2(0,-1,2) -3(-2,0,1) = (0,-2,4) - (-6,0,3) = (6,-2,1)$. Ao somar os vetores obtidos nos três casos, o resultado também será zero. A diferença é que terá feito mais cálculos, sem necessidade. De fato, não é preciso calcular numericamente nem mesmo os produtos escalares, pois o mesmo raciocínio funciona se desenvolver a soma utilizando as expressões $\vec{u} \cdot \vec{v}$, $\vec{u} \cdot \vec{w}$ e $\vec{v} \cdot \vec{w}$ diretamente, sem troca-las por números (verifique!).

\item \textit{ Determine as equações reduzidas (com variável independente $y$) da reta $r$ que passa por $A = (1,2,-2)$ e que é perpendicular ao plano $-\frac{x}{2} + y - 3z - 5 = 0$.}

O plano em questão é ortogonal ao vetor $\vec{n} = (-\frac{1}{2}, 1, -3)$ (e também ao seu múltiplo $\vec{n}_1 = 2 \vec{n} = (-1, 2, -6)$, caso prefira evitar frações). Utilizando o ponto $A$ e o vetor $\vec{n}$, conclui-se que a reta $r$ tem as seguintes equações paramétricas (também podiam ser utilizadas as equações simétricas):
\[
\begin{cases}
x = 1 - \frac{t}{2}\\
y = 2 + t\\
z = -2 - 3t
\end{cases}
\]
Da segunda equação, resulta que $t = y - 2$, e então as demais equações podem ser reescritas assim:
\[
\begin{cases}
x = 1 - \frac{1}{2}(y - 2)\\
z = -2 - 3(y - 2).
\end{cases}
\]
Portanto, as equações reduzidas são
\[
\begin{cases}
x = -\frac{y}{2} +2 \\
z = - 3y + 4.
\end{cases}
\]

\item \textit{ Encontre equações paramétricas da reta $r$ dada pela interseção do plano $-3x + 6y - 4z + 12 = 0$ com o plano $-x + 2y + 6z + 4 = 0$. }

Todos os pontos $(x, y, z) \in r$ devem satisfazer simultaneamente às equações dos planos, isto é,
\[
\systeme{
-3x + 6y - 4z + 12 = 0,
 -x + 2y + 6z +  4 = 0}
\]
Da segunda equação, resulta que $x = 2y + 6z +  4$, e essa expressão pode ser substituída na primeira equação para obter que $-3(2y +  6z +  4) + 6y - 4z + 12 = 0$. Simplificando, conclui-se que $- 22z = 0$, o que significa que $z = 0$.

Com isso, a segunda equação do sistema pode ser reescrita como $-x + 2y + 4 = 0$, o que implica que $x = 2y + 4$. Assim, a reta $r$ pode ser descrita pelas equações reduzidas (com variável independente $y$)
\[
\systeme*{
x = 2y + 4,
z = 0.
}
\]
ou equivalentemente, pelas equações paramétricas
\[
\systeme*{
x = 2t + 4,
y = t,
z = 0
}
\]
que são obtidas ao considerar o ponto $A = (4, 0, 0) \in r$ e o vetor diretor $\vec{v} = (2, 1, 0)$ de $r$.

\item \textit{ Calcule o ângulo entre a reta $r: x+2 = \dfrac{y-3}{-2} = z$ e a reta $s : \begin{cases}
x = -y -1 \\
z = 2.
\end{cases}$ }

O vetor $\vec{u} = (1,-2,1)$ é um vetor diretor da reta $r$, e o vetor $\vec{v} = (-2,1,2) - (-1,0,2) = (-1,1,0)$ é um vetor diretor de $s$. Então, o ângulo $\theta$ entre $r$ e $s$ é dado por
\[
\cos(\theta)
= \frac{|\vec{u} \cdot \vec{v}|}{\norm{u} \cdot \norm{v}}
= \frac{|1 \cdot (-1) + (-2) \cdot 1 + 1 \cdot 0|}
       {\sqrt{1 + 4 + 1} \cdot \sqrt{1+1+0}}
= \frac{|-3|}{\sqrt{6} \cdot \sqrt{2}}
= \frac{3}{2\sqrt{3}}
= \frac{\sqrt{3}}{2}.
\]
Logo, $\theta = \operatorname{arccos}(\frac{\sqrt{3}}{2}) = \frac{\pi}{6}$, que corresponde a um ângulo de $30\degree$.

\item \textit{ Forneça uma equação para o plano que passa pelo ponto médio de $A = (2, 1, 3)$ e $B = (0, 3, -9)$, e que além disso contém a reta $x = y = z$. }

O ponto médio de $A$ e $B$ é $M = \left(\frac{2+0}{2}, \frac{1+3}{2}, \frac{3-9}{2}\right) = (1,2,-3)$. O vetor $\vec{n}$, normal do plano $\pi$ procurado, será ortogonal a um vetor diretor $\vec{v} = (1,1,1)$ da reta. Além disso, $\vec{n}$ também será ortogonal a qualquer vetor $\vec{u}$ que tenha origem em um ponto arbitrário da reta (por exemplo, $(1,1,1)$), e extremidade em $M$. Deste modo, podemos obter um vetor $\vec{n}$ fazendo o produto vetorial entre $\vec{v}$ e $\vec{u} = M - (1,1,1) = (0,1,-4)$:

\[
\vec{v} \times \vec{u} =
\begin{vmatrix}
\vec{i} & \vec{j} & \vec{k} \\
1 & 1 & 1 \\
0 & 1 & -4
\end{vmatrix}
= (-4 -1) \vec{i} - (-4 - 0) \vec{j} + (1 - 0) \vec{k}
= (-5, 4, 1).
\]
Logo, a equação geral do plano tem a forma $-5x + 4y + z + d = 0$, e como $M \in \pi$, devemos ter $-5 \cdot 1 + 4\cdot 2 + (-3) + d = 0$, o que implica que $d = 0$. Portanto o plano tem equação geral $-5x + 4y + z = 0$.

\item \textit{ Seja $s$ a reta $\frac{x + 2}{2} = \frac{y - 3}{-3} = z$. Dê um exemplo (com equações paramétricas) de uma reta $r_1 // s$, de uma reta $r_2$ concorrente com $s$ e de uma reta $r_3$ reversa com $s$ (explique suas escolhas). }

A reta $s$ tem vetor diretor $\vec{v} = (2,-3,1)$.
\begin{enumerate}
\item Para obter uma reta $r_1 // s$, basta utilizar o mesmo vetor diretor $\vec{v}$ (ou um múltiplo dele), e qualquer ponto que não pertença a $s$ (pois os vetores diretores de retas paralelas são colineares, e elas só têm algum ponto em comum se forem coincidentes). Se escolhermos, por exemplo, $A = (0,0,0)$, é claro que suas coordenadas não satisfazem as equações simétricas de $s$, e portanto $A \not\in s$. As equações paramétricas da reta $r_1$ que passa por $A$ e que tem a direção de $\vec{v}$ são:
\[
\systeme*{
x = 2t,
y = -3t,
z = t
}
\]

\item Para qualquer ponto $B \in s$ e qualquer vetor $\vec{u}$ que não seja colinear com $\vec{v}$, a reta que passa por $B$ e que tem a direção de $\vec{u}$ é concorrente com $s$. Assim, como $B = (-2,3,0) \in s$ e $\vec{u} = (1,0,0)$ não é múltiplo de $\vec{v}$, as equações paramétricas
\[
\systeme*{
x = -2 + t,
y = 3,
z = 0
}
\]
representam uma reta $r_2$, entre as infinitas retas que são concorrentes com $s$.

\item Se no item anterior trocarmos o ponto $B \in s$ por um ponto $C$ que não pertence nem a $s$ nem a $r_2$, obteremos uma reta $r_3$ que não é paralela a $s$, e que também não possui um ponto em comum com $s$, ou seja, $r_3$ e $s$ serão retas reversas. A título de exemplo, pode-se considerar $C = A = (0,0,0)$, que não pertence a $s$ nem a $r_2$, e obter as equações paramétricas
\[
\systeme*{
x = t,
y = 0,
z = 0
}
\]
que representam o eixo $Ox$.
\end{enumerate}
\end{enumerate}
\end{document}
