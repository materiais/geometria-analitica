\documentclass[12pt,a4paper]{article}
\usepackage{cmap} % Makes the PDF copiable. See http://tex.stackexchange.com/a/64198/25761
\usepackage[T1]{fontenc}
\usepackage[brazil]{babel}
\usepackage[utf8]{inputenc}
\usepackage{amsmath}
\usepackage{amsfonts}
\usepackage{amssymb}
\usepackage{amsthm}
\usepackage{textcomp} % \degree
\usepackage{gensymb} % \degree
\usepackage[usenames,svgnames,dvipsnames]{xcolor}
\usepackage{hyperref}
\usepackage{graphicx}
\usepackage[margin=2cm]{geometry}

\hypersetup{
    colorlinks = true,
    allcolors = {blue}
}

% TODO: Consider using exsheets
% http://linorg.usp.br/CTAN/macros/latex/contrib/exsheets/exsheets_en.pdf
%
% http://ctan.org/tex-archive/macros/latex/contrib/exercise/
% Options: answerdelayed,lastexercise,noanswer
\usepackage[answerdelayed,lastexercise]{exercise}

\addto\captionsbrazil{%
\def\listexercisename{Lista de exerc\'icios}%
\def\ExerciseName{Exerc\'icio}%
\def\AnswerName{Solu\c{c}\~ao do exerc\'icio}%
\def\ExerciseListName{Ex.}%
\def\AnswerListName{Solu\c{c}\~ao}%
\def\ExePartName{Parte}%
\def\ArticleOf{de\ }%
}

\renewcommand{\ExerciseHeaderTitle}{(\ExerciseTitle)\ }
\renewcommand{\ExerciseListHeader}{%\ExerciseHeaderDifficulty%
\textbf{%\ExerciseListName\
\ExerciseHeaderNB.\ %
%\ --- \ 
\ExerciseHeaderTitle}%
%\ExerciseHeaderOrigin
\ignorespaces}
\renewcommand{\AnswerListHeader}{\textbf{\ExerciseHeaderNB.\ (\AnswerListName)\ }}

\newcommand{\fixme}{{\color{red}(...)}}

\renewcommand{\theenumi}{\alph{enumi}}
\renewcommand\labelenumi{(\theenumi) }

\newcommand*\tipo{Prova II}
\newcommand*\turma{PRO112-01U}
\newcommand*\disciplina{GAN0001}
\newcommand*\eu{Helder G. G. de Lima}
\newcommand*\data{29/09/2016}

\author{\eu}
\title{\tipo - \disciplina}
\date{\data}

\begin{document}
\thispagestyle{empty}
\newgeometry{margin=2cm,bottom=0.5cm}
\begin{center}
\includegraphics[width=9.0cm]{marca} \\
\textbf{\tipo\ (\disciplina / \turma)} \\
Prof. \eu\footnote{
Este é um material de acesso livre distribuído sob os termos da licença \href{https://creativecommons.org/licenses/by-sa/4.0/deed.pt_BR}{Creative Commons BY-SA 4.0}.}
\end{center}

\noindent Nome do(a) aluno(a): \underline{\hspace{9,7cm}} Data: \underline{\data}

%\section*{Instruções}
\begin{center}\fbox{
\begin{minipage}{14cm}

{\footnotesize
\begin{itemize}
\renewcommand{\theenumi}{\Roman{enumi}}
\item Identifique-se em todas as folhas.
\item Mantenha o celular e os demais equipamentos eletrônicos desligados durante a prova.
\item Escolha os itens a resolver de modo a totalizar apenas 10 pontos.
%\rule{3cm}{0.01cm}
\end{itemize}
}

\end{minipage}
}
\end{center}

%\section*{Questões}
\begin{ExerciseList}
\Exercise%[title={6,0}]
Considere as seguintes retas:
\[
r_1: \begin{cases}
x=-y+1\\
z=-\frac{3}{2}y+2,
\end{cases}
\qquad
r_2: \begin{cases}
x=-4t\\
y=1+4t\\
z=3-6t
\end{cases}
\qquad\text{e}\qquad
r_3: \dfrac{x-6}{4} = \dfrac{y+3}{-2} = \dfrac{z-9}{3}.
\]
\begin{enumerate}
\item \textbf{(1,0)} Em que ponto cada reta intersecta o plano $xOy$?
\item \textbf{(1,0)} Que reta é paralela a uma das outras retas? Qual é a distância entre elas?
\item \textbf{(2,0)} Há duas retas concorrentes? Se houver, em que ponto se intersectam?
\item \textbf{(2,0)} Duas das retas são reversas? Em caso afirmativo, qual é a distância que as separa?
\end{enumerate}
\Answer \fixme

\Exercise%[title={6,0}]
Considerando os planos $\pi_1: 3x + 6y - 3z + 15 = 0$ e $\pi_2: -x + y - 2z - 2 = 0$, determine:
\begin{enumerate}
\item \textbf{(1,0)} Qual dos planos contém $Q(-1,5,2)$, e qual é a distância entre $Q$ e o outro plano.
\item \textbf{(1,0)} O ângulo entre $\pi_1$ e $\pi_2$.
\item \textbf{(1,0)} As equações reduzidas da reta $s_1$ definida pela interseção de $\pi_1$ com o plano $xOz$.
\item \textbf{(1,0)} As equações paramétricas da reta $s_2$ definida pela interseção de $\pi_2$ com o plano $xOz$.
\item \textbf{(1,0)} O ângulo entre as retas $s_1$ e $s_2$ obtidas anteriormente.
\item \textbf{(2,0)} O volume do tetraedro limitado pelo plano $\pi_1$ e os $3$ planos coordenados.
\end{enumerate}
\Answer \fixme
\end{ExerciseList}

\begin{center}
BOA PROVA!
\end{center}

%\newpage
\restoregeometry
%\section*{Respostas}
%\shipoutAnswer
\end{document}
