\documentclass[12pt,a4paper]{article}
\usepackage{cmap} % Makes the PDF copiable. See http://tex.stackexchange.com/a/64198/25761
\usepackage[T1]{fontenc}
\usepackage[brazil]{babel}
\usepackage[utf8]{inputenc}
\usepackage{amsmath}
\usepackage{amsfonts}
\usepackage{amssymb}
\usepackage{amsthm}
\usepackage{textcomp} % \degree
\usepackage{gensymb} % \degree
\usepackage[usenames,svgnames,dvipsnames]{xcolor}
\usepackage{hyperref}
\usepackage{graphicx}
\usepackage[margin=2cm]{geometry}

\hypersetup{
    colorlinks = true,
    allcolors = {blue}
}

% TODO: Consider using exsheets
% http://linorg.usp.br/CTAN/macros/latex/contrib/exsheets/exsheets_en.pdf
%
% http://ctan.org/tex-archive/macros/latex/contrib/exercise/
% Options: answerdelayed,lastexercise,noanswer
\usepackage[answerdelayed,lastexercise]{exercise}

\addto\captionsbrazil{%
\def\listexercisename{Lista de exerc\'icios}%
\def\ExerciseName{Exerc\'icio}%
\def\AnswerName{Solu\c{c}\~ao do exerc\'icio}%
\def\ExerciseListName{Ex.}%
\def\AnswerListName{Solu\c{c}\~ao}%
\def\ExePartName{Parte}%
\def\ArticleOf{de\ }%
}

\renewcommand{\ExerciseHeaderTitle}{(\ExerciseTitle)\ }
\renewcommand{\ExerciseListHeader}{%\ExerciseHeaderDifficulty%
\textbf{%\ExerciseListName\
\ExerciseHeaderNB.\ %
%\ --- \ 
\ExerciseHeaderTitle}%
%\ExerciseHeaderOrigin
\ignorespaces}
\renewcommand{\AnswerListHeader}{\textbf{\ExerciseHeaderNB.\ (\AnswerListName)\ }}

\newcommand{\fixme}{{\color{red}(...)}}
\newcommand*\sen{\operatorname{sen}}

\renewcommand{\theenumi}{\alph{enumi}}
\renewcommand\labelenumi{(\theenumi) }

\newcommand*\tipo{Prova III}
\newcommand*\turma{PRO112-01U}
\newcommand*\disciplina{GAN0001}
\newcommand*\eu{Helder G. G. de Lima}
\newcommand*\data{01/11/2016}

\author{\eu}
\title{\tipo - \disciplina}
\date{\data}

\begin{document}
\thispagestyle{empty}
\newgeometry{margin=2cm,bottom=0.5cm}
\begin{center}
\includegraphics[width=9.0cm]{marca} \\
\textbf{\tipo\ (\disciplina / \turma)} \\
Prof. \eu\footnote{
Este é um material de acesso livre distribuído sob os termos da licença \href{https://creativecommons.org/licenses/by-sa/4.0/deed.pt_BR}{Creative Commons BY-SA 4.0}.}
\end{center}

\noindent Nome do(a) aluno(a): \underline{\hspace{9,7cm}} Data: \underline{\data}

%\section*{Instruções}
\begin{center}\fbox{
\begin{minipage}{14cm}

{\footnotesize
\begin{itemize}
\renewcommand{\theenumi}{\Roman{enumi}}
\item Identifique-se em todas as folhas.
\item Mantenha o celular e os demais equipamentos eletrônicos desligados durante a prova.
\item Escolha os itens a resolver de modo a totalizar até 10,0 pontos.
\end{itemize}
}

\end{minipage}
}
\end{center}

%\section*{Questões}
\begin{ExerciseList}
\Exercise[title={2,0}] Uma elipse possui o centro em $C=(-3,1)$, um foco em $F_1=(-3,-2)$ e um vértice em $A_1=(-3,-3)$. Determine a equação da elipse, sua excentricidade e o comprimento de cada um de seus eixos.
\Answer \fixme

\Exercise%[title={2,0}]
Uma hipérbole que passa pelo ponto $A=(4,0)$ possui vértices $A_1=(0,-2)$ e $A_2=(2, -2)$.
\begin{enumerate}
\item \textbf{(1,0)} Determine a equação reduzida da hipérbole.
\item \textbf{(1,0)} Obtenha as equações das assíntotas, e utilize-as para esboçar a hipérbole.
\end{enumerate}
\Answer \fixme

\Exercise%[title={2,0}]
Se uma cônica é representada pela equação $2x^2 - 4x + \dfrac{9}{8}y^2 + 9y + 2 = 0$, obtenha:
\begin{enumerate}
\item \textbf{(0,4)} A equação reduzida da cônica.
\item \textbf{(0,8)} A excentricidade, as coordenadas do centro, foco(s) e vértice(s), e as equações das assíntotas ou da diretriz (conforme o tipo de curva encontrado).
\item \textbf{(0,8)} Um esboço da curva coerente com os dados obtidos acima.
\end{enumerate}
\Answer \fixme

\Exercise[title={2,0}] Esboce a curva $r=6\cos(2\theta)$, exibindo as coordenadas polares de 7 pontos distintos e explicando se há algum tipo de simetria.
\Answer \fixme

\Exercise[title={2,0}] Esboce a curva $r=3-3\cos \left(\theta \right)$, exibindo as coordenadas polares de 7 dos seus pontos e explicando se há algum tipo de simetria.
\Answer \fixme

\Exercise%[title={2,0}]
Considere a curva cujos pontos $P=(r, \theta)$, em coordenadas polares, satisfazem $r = \dfrac{3}{1-\sen{\theta}}$.
\begin{enumerate}
\item \textbf{(1,0)} Obtenha uma equação equivalente (simplificada) em coordenadas cartesianas $(x,y)$.
\item \textbf{(1,0)} Esboce a curva, exibindo as coordenadas polares de pelo menos 5 pontos utilizados.
\end{enumerate}
\Answer \fixme
\end{ExerciseList}

\begin{center}
BOA PROVA!
\end{center}

%\newpage
\restoregeometry
%\section*{Respostas}
%\shipoutAnswer
\end{document}
